\subsection{Balls detection}

To detect balls, Michele proposed a multi-step preprocessing approach. Initially, the table region was isolated by constructing a polygon using its corners and a color-based mask is generated. Subsequently, pixels outside the table were nullified, and k-means clustering was applied to the image. The resulting clusters were converted to gray-scale for Hough Circle Transform application.
Circle parameters, such as radius and center color, were analyzed to identify potential ball regions. By calculating the mean radius of in-table circles with center not selected by the color mask, a radius range was established. Circles within this radius range were considered for further analysis. Ball classification involved creating a circular mask, computing the gray-scale histogram, and excluding background pixels from the values of the histogram. Peak values in the histogram were used to differentiate between striped and solid balls, while HSV color space analysis is used to distinguish white and black balls. 
After finding the balls, the team identified an optimization opportunity. Since there's only one white ball and one black ball, Michele implemented non-maxima suppression for white and black balls independently, in order to improve performance.
The result of the detection process is then used to segment the balls.