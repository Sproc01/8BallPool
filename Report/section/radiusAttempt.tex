\subsubsection{Attempt to find the ball radius relative to the distance and perspective of the camera with respect to the table}
To try to increase the performance of the ball detection, Michela had the idea of to computing an interval of values for the ball radius relative to the pixels of the image and the position of the camera with respect to the table; this would have been used in the \texttt{HoughCircles}.
For that purpose, she wrote the method \texttt{radiusInterval}.

This method starts by computing the mean radius value by using a proportion between the diagonal of the table in pixels and the dimensions of the diagonal of the table and the balls in centimeters (such dimensions in centimeters are not standard, so an average value has been selected).
\begin{equation}
	mean\_radius = \frac{ball\_radius\_cm}{table\_diagonal\_cm} \times longest\_diagonal\_px
\end{equation}

Then, a percentage of the slope between the camera direction and the table has been computed, by using one of the angles ($<90^{\circ}$) that the table detection provides; this angle is compared with the $\frac{\pi}{2}$ angle, and a value between $0$ and $1$ is computed:
\begin{equation}
	percentage\_slope = 1 - \frac{minimum\_angle}{\pi}
\end{equation}

\begin{itemize}
	\item If the value is 1, then the camera is parallel to the table;
	\item If the value is 0, then the camera is perpendicular to the table;
	\item Otherwise it is a value between 0 and 1, which indicates the percentage of slope between the camera and the table; for example, if the value is 0.5, then the camera is about 45° from the table.
\end{itemize}

To compute the final interval, the minimum and maximum values are computed by subtracting and incrementing a value, which increases with the percentage of slope (more the slope, more the variance) by multiplying the percentage of slope with the mean radius previously computed, and a precision value is added due to some other variables in the images.
\begin{equation}
	min\_radius = mean\_radius - mean\_radius \times percentage\_slope - precision
\end{equation}
\begin{equation}
	max\_radius = mean\_radius + mean\_radius \times percentage\_slope + precision
\end{equation}
