\subsection{Video creation}
To create the output video we take the minimap and the current frame and Michele decide to do as follows:
At the beginning the function compute two values: the scaling factor for the minimap and the offset
(the first row where the minimap is placed). For the scaling factor Michele thought that
the minimap should have 0.3 of the total columns of the frame, so the scaling factor is computed as follows:
\begin{equation}
	\text{scaling\_factor} = \frac{0.3 \times \text{frame.cols}}{\text{minimap\_with\_balls.cols}}
\end{equation}
Instead for the offset, Michele thought that the minimap should be placed at the bottom right of the frame
and it must be attached to the bottom border so first a factor is computed as follows:
\begin{equation}
	\text{percentage} = \frac{(\text{frame.rows} - \text{scaling\_factor} \times \text{minimap\_with\_balls.rows})}{\text{frame.rows}}
\end{equation}
Then the offset is computed as follows:
\begin{equation}
	\text{offset} = \text{percentage} \times \text{frame.rows}
\end{equation}
After that the minimap is resized and placed in bottom right corner of the output frame that
is saved in the output video.
To avoid creating the output video file and then run in some exception resulting in a not readable file,
Alberto thought about using a temporary file and then rename it to the output file only at the end where
it is sure that the file is readable.
